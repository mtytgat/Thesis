\renewcommand\evenpagerightmark{{\scshape\small Nederlandse Samenvatting}}
\renewcommand\oddpageleftmark{{\scshape\small Dutch Summary}}

\hyphenation{}

\chapter[Nederlandse samenvatting]{Nederlandse samenvatting\\\makebox[2.82in]{--Dutch Summary--}}
\label{samenvatting}

	De eerste stap in de upgrade van de {\it Large Hadron Collider} (LHC) naar de {\it High Luminosity LHC} (HL-LHC) werd aangevat eind 2018, bij het begin van de zogenaamde {\it Second Long Shutdown} (LS2), waarbij de deeltjesversneller en de bijhorende experimenten voor 2 jaar stilgelegd werden voor onderhoudswerken en aanpassingen. Het doel hierbij is de luminositeit van de versneller met een factor 10 te verhogen om zo het potentieel op te drijven voor de ontdekking van nieuwe fysica die niet door het huidige Standaardmodel (SM) van de deeltjesfysica beschreven wordt. Vele uitbreidingen van het SM voorspellen het bestaan van nieuwe deeltjes zoals bijvoorbeeld {\it Heavy Stable Charged Particles} (HSCPs) waarbij, binnen de context van het Compacte Muon Solenoïde (CMS) experiment, het muonsysteem een belangrijke rol zou kunnen spelen om deze deeltjes te identificeren. Een toename in de ogenblikkelijke luminositeit van de versneller zal aanleiding geven tot een verhoging van zowel de achtergrondruis alsook de stralingsniveaus waaraan de detectoren blootgesteld zullen worden. Bovendien, hoewel de huidige detectoren ontworpen werden voor 10 jaar bedrijf aan standaard LHC condities, wordt er nu verwacht dat ze nog eens minstens 10 jaar langer operationeel blijven tijdens de HL-LHC met sterk verhoogde luminositeit. Daarom werd er een uitgebreid R&D programma gelanceerd voor de upgrade van alle CMS muonsubsystemen naar de HL-LHC fase toe.\\ 
	Een eerste gegeven in deze upgrade van het muonsysteem is het feit dat de detectoren gecertifieerd moeten worden voor de HL-LHC periode. Daarnaast zal het muonsysteem in de voorwaartse detectorregio, d.w.z. dichtbij de LHC deeltjesbundel, ook uitgebreid worden met nieuwe detectoren, inclusief twee nieuwe stations uitgerust met verbeterde Resistive Plate Chambers (RPCs), de zogenaamde RE3/1 en RE4/1 stations. De bedoeling hiervan is om een hoge kwaliteit van de muontrigger te verzekeren door effecten ten gevolge van achtergrond te minimaliseren en door de redundantie van het systeem te verhogen. Dit zal eveneens bevorderlijk zal zijn voor de reconstructie van de muonsporen doorheen de CMS detector. Een upgrade zal ook plaatsvinden op niveau van het {\it Link Board} systeem dat de front-end electronica (FEE) van de CMS RPCs verbindt met de triggerprocessoren, om zo de robuustheid van het systeem te verbeteren en ook beter gebruik te maken van de effectieve tijdsresolutie van de Resistive Plate Chambers (van de orde \SI{1}{ns}). De CMS RPC upgrade omvat zodoende een uitgebreid onderzoeksprogramma. De CMS RPC collaboratie kwam na een aantal jaren van R&D tot een reeks passende oplossingen die zullen uitgevoerd worden in het vooruitzicht van de HL-LHC. Hoewel de consolidatie van de huidige CMS RPC-infrastructuur en de certificatie van nieuwe technologieën nog steeds verder loopt, is de toekomst van CMS wat de RPCs betreft ondertussen zeer duidelijk. Om dit punt te bereiken is mijn bijdrage tijdens de R\&D-fase van het RPC-upgradeprogramma cruciaal geweest op twee vlakken: (i) de consolidatie van de huidige detectoren, en (ii) de keuze van de FEE waarmee de verbeterde RPCs (ofwel iRPCs) zullen uitgerust worden. Bij elke stap in het onderzoek speelde ik een belangrijke rol zowel in het opzetten van de testexperimenten als in het vergaren en analyseren van de metingen.\\
	De certificatie van de nieuw voorgestelde technologieën die beantwoorden aan de noden van CMS voor de RE3/1- en RE4/1-regio's van de muon endcap disks, en van het huidige RPC-systeem wordt momenteel uitgevoerd in de nieuwe {\it Gamma Irradiation Facility} (GIF++) op CERN, waar een stralingsbunker rondom een nieuwe muonenbundel werd gebouwd. Deze faciliteit beschikt over een cesiumbron van \SI{14}{TBq}. Het doel van deze nieuwe faciliteit is het simuleren van de HL-LHC condities en het uitvoeren van verscheidene lange termijn studies binnen de context van het HL-LHC upgrade programma. Het prototype van de nieuwe iRPCs voor CMS zijn trapeziumvormige {\it double-gap} kamers bestaande uit twee \SI{1.4}{mm} dikke gasruimtes gevormd door even dikke {\it High-Pressure Laminate} (HPL) elektroden. Het verlagen van de dikte van de elektroden en gasruimtes t.o.v. het huidige RPC systeem werd gedreven door de lagere versterking of {\it gain} van zulke detectoren bij gelijkaardige elektrisch veldsterktes, wat de levensduur van deze detectoren ten goede komt. De uitlezing van de RPC-signalen zal gebeuren via een {\it Printed Circuit Board} (PCB) dat 96 longitudinale strips bevat met een onderlinge afstand die varieert tussen de \SI{4}{mm} aan het smalle uiteinde en \SI{8}{mm} aan de brede zijde van de trapeziumvorm. Dit ontwerp definieert de basisversie van wat er uiteindelijk in CMS geïnstalleerd zal worden.\vspace*{5mm}
	
	Mijn bijdrage aan de CMS upgrade was het valideren van twee mogelijke FEE technologieën: (i) een voorversterker met laag ruisniveau die gebruik maakt van SiGe-technologie ontwikkeld door INFN Tor Vergata, en (ii) FEE gebaseerd op de PETIROC ASIC die eveneens gebruik maakt van SiGe-technologie ontwikkeld door de OMEGA collaboratie en vooral gebruikt wordt voor tijdsgerelateerde toepassingen. Om de veroudering van de geïnstalleerde detectoren in de voorwaartse richting nabij de LHC bundellijn waar het stralingsniveau relatief hoog is, tegen te gaan, zal een verlaging van de benodigde ladingsdepositie binnenin de gaskamer om een meetbaar signaal te genereren zeker helpen. Dit kan bekomen worden door gebruik te maken van meer gevoelige elektronica met versterkers met een lagere drempelwaarde voor elektrische lading, d.w.z. \SI{10}{fC} in plaats van \SI{140}{fC} zoals op het huidige CMS RPC Front-end Board (FEB).\\
	In een eerste stap werd de INFN Tor Vergata voorversterker tweemaal getest en vergeleken met de huidige CMS FEE. Bij een eerste test werd de voorversterker rechtstreeks verbonden met vier uitleesstrips van een reserve-RPC, en werd het uitgangssignaal gedigitaliseerd door middel van een discriminator. De voorversterker werd gebruikt met een lage drempelwaarde van \SI{3}{fC} voor gedetecteerde lading, en resulteerde in een verlaging van de RPC-werkspanning van \SI{475}{V} ten opzichte van een RPC uitgerust met de huidige CMS-elektronica. Om de vergelijking van de INFN-technologie met de huidige CMS FEB te verbeteren werd een nieuwe versie van de FEB zonder voorversterker ontworpen die dan met de voorversterker verbonden werd. Dit liet ons toe een directe vergelijking te maken tussen de nieuwe voorversterker met lage ruis en de huidige CMS RPC FEB. De elektronica werd wederom gemonteerd op een reserve-PRC om de voorgaande test te herhalen. De verlaging van de geobserveerde RPC werkspanning liep op tot \SI{410}{V} bij een drempelwaarde van \SI{5}{fC}, wat consistent is met mijn eerste resultaat.\\
	De eerste versie van de FEE ontworpen door OMEGA werd getest op \'e\'en enkele reserve RPC HPL gaskamer uit de oude RE4-productie. Om een vergelijking te maken werd het geheel in een RPC-behuizing gemonteerd en werd de detector ook met de standaard CMS FEB in single-gap modus bedreven. Het resultaat toonde aan dat de HARDROC2 ASIC een reductie in de RPC werkspanning van bijna \SI{500}{V} kan teweegbrengen in single-gap modus bij een ladingsgevoeligheid van \SI{121.4}{fC}, hetgeen vergelijkbaar is met de drempelwaarde van \SI{146}{fC} bij de standaard CMS FEE, terwijl de ruiswaarde lager dan \SIrate{0.1} blijft. Diezelfde OMEGA technologie werd ook eerder al gecertifieerd door andere experimenten die gebruik maakten van detectoren zoals scintillatoren of andere RPC types. De OMEGA FEE biedt eveneens de mogelijkheid tot een tweedimensionale uitlezing van de RPC strips, hetgeen de spatiale resolutie van de detector in de radiale richting sterk zou verbeteren t.o.v. het huidige systeem. Er werd aangetoond dat gebruik maken van de goede tijdresolutie van deze elektronica en van het uitlezen van de iRPC uitleesstrips aan beide uiteinden zal resulteren in een spatiale resolutie van ongeveer \SI{2}{cm} langsheen de richting van de strips, wat de bijdrage van de RPCs aan de muon spoorreconstructie enorm zal verhogen ten opzichte van het huidige CMS muonsysteem.\\
	Verschillende FEE prototypes aangepast aan de vereisten voor de CMS RPC werden geproduceerd om deze in meer detail te testen, d.w.z. zowel deze afgeleid van de OMEGA PETIROC technologie alsook van het INFN Tor Vergata alternatief. Ter vergelijking werden ook testen met kosmische muonen zonder bijkomende bestraling uitgevoerd op een iRPC prototype gebruik makende van de huidige CMS FEBs. De detector bereikte zijn werkspanning op \SIerror{7383}{70}{V} met een efficiëntie van 97\% bij een drempelwaarde van \SI{133}{fC}. De gemiddelde grootte van de muonclusters werd bepaald op \numerror{2.4}{0.1} strips, terwijl en het ruisniveau van de orde \Ord{-1}\,\sirate was.\\
	Tot nu toe werden al twee versies (FEBv0 en FEBv1) van de FEE gebaseerd op de PETIROC ASIC gekoppeld aan een FPGA getest, terwijl FEBv2 nog in voorbereiding is. Dit type FEE is efficiënt ontworpen om de 96 strips langs beide uiteinden uit te lezen, wat tweedimensionale plaatsinformatie geeft over de muon hits. FEBv0 werd gebruikt met een drempelwaarde van \SI{100}{fC} terwijl FEBv1 aan een lagere waarde van \SI{50}{fC} kon bedreven worden dankzij een PETIROC versie met een verlaagd cross-talk niveau. Zonder bijkomende bestraling bereikte het prototype voorzien van FEBv1 een efficiëntie van 99\% bij een werkspanning van \SI{7250}{V}. Met een achtergrondruis van \SIkrate{2} liep de werkspanning op tot \SI{7340}{V} bij een efficiëntie van 95\%. Echter, zowel de Cyclone V FPGA en PETIROC2B ASIC hadden last van problemen gerelateerd aan stralingshardheid. FEBv2 zal dan ook een nieuwe versie van de Cyclone V FPGA bevatten, die speciaal ontworpen is voor toepassingen in omgevingen met hoge stralingsdichtheid. De nieuwe PETIROC2C wordt momenteel getest in de Louvain-la-Neuve (LLN) neutronenbundel met een fluentie van \Ord{14}\,\si{pC/cm^2}, wat vijf maal de verwachte waarde bij CMS is.\\
	Het INFN FEE testboard bevat acht SiGe voorversterkers, alsook twee discriminator-ASICs. Aldus werden twaalf van deze FEE boards geïntegreerd op een uitlees PCB. De detector werd bedreven aan een drempelwaarde van \SI{5}{fC} en werd blootgesteld aan een achtergrond stralingsniveau tot \SIkrate{4}, waarbij hij een efficiëntie aanhield van meer dan 92\%. Zonder bijkomende bestraling had hetzelfde prototype een efficiënte die hoger lag dan 99\% voor een werkspanning lager dan \SI{7000}{V}. De gemiddelde muonclustergrootte werd bepaald op \numerror{3.4}{0.1}, wat als gevolg van de verhoogde gevoeligheid van de nieuwe FEE hoger is in vergelijking met de CMS FEB. Met een stralingsniveau van \SIkrate{2} lag de efficiëntie op 97\% en de werkspanning op \SI{7250}{V}. Onder invloed van de achtergrondstraling daalde de muonclustergrootte tot \numerror{2.6}{0.1}.\vspace*{5mm}
	
	Mijn grootste bijdrage aan de CMS RPC-upgrade was gerelateerd aan de langlevendheidstudie en de certificatie van het huidige systeem voor de HL-LHC. Volgens extrapolaties van de huidige CMS-metingen zouden de detectoren in het bestaande RPC systeem gecertificeerd moeten worden voor een achtergrondstraling van \SIrate{600} en een totale geïntegreerde lading van \SI{840}{mC/cm^2}. Wat overeenstemt met driemaal de ergste achtergrondruis zoals geëxtrapoleerd uit CMS-data uit het RPC-systeem. Ik sloot mij aan bij de voorafgaande studie die plaatsvond bij de oude GIF van het CERN, waarbij een cesiumbron van 494 GBq beschikbaar was. Een eerste opstelling met een enkel reserve-exemplaar van een CMS RPC vormde een gelegenheid om eerste versies van datacollectie-, datakwaliteitmonitorings- en data-analysetools te ontwikkelen. De plaatkamer werd geïnstalleerd in een bunker voor de radioactieve bron om vervolgens bestraald te worden met een achtergrondruis van \SIrate{600}. De performantie van de detector werd getest met een scintillator-gebaseerde kosmische muontelescoop als trigger. De resultaten suggereerden dat de performantie van de detector gedaald was tot 80\% efficiëntie met een werkzame spanning verhoogd tot \SI{1000}{V} bij \SIrate{600}.\\
	Een grootschaliger experiment werd vervolgens ontworpen aan de GIF++. Deze nieuwe opstelling maakte gebruikt van twee reserve-RPCs van het RE2/2 type, geproduceerd in 2007 en twee van het RE4/2 type uit 2013. Eén plaatkamer van elk type diende als referentie terwijl de tweede diende voor studies naar langdurigheid onder straling. Op moment van schrijven is respectievelijk 74 en 40\% van het geplande programma uitgevoerd sinds 2016. De opvolging van de stroomdichtheid in de plaatkamer en van de achtergrondstraling vertoont een sterke correlatie met de temperatuur aan de GIF++. Wanneer dit in rekening wordt gebracht, tonen de stroomdichtheid en de achtergrondruis gemeten in de detector een daling ten opzichte van de referentiedetectors. Dit effect lijkt gecorreleerd te zijn met een toename aan resistiviteit in de bestraalde detectors met een factor 2 ten opzichte van de referentiedetectors. Echter, de schommelingen in de stroomdichtheid en de achtergrondruis tonen een correlatie met de relatieve luchtvochtigheid in het gasmengsel. Het effect lijkt dus reversibel door de plaatkamers aan een hogere relatieve luchtvochtigheid te gebruiken.\\
	Geen verlies aan performantie was merkbaar bij een achtergrondruis van \SIrate{600}. Alle detectors geven blijk van een efficiëntie boven 94\% tot op de vereiste ruisniveaus. De opgemeten verschuiving in werkzame spanning van 100 V voor de RE2 RPCs en van 300 tot \SI{500}{V} voor de RE4 RPCs is consistent met het verschil in de elektroderesistiviteit van beide detectortypes. De RE2 detectors hebben een resistiviteit in het interval tussen 1 en \Sci{3}{10}\,\si{\ohm\cdot cm} terwijl die van de nieuwere RE4 ligt tussen 0.7 en \Sci{2}{11}\,\si{\ohm\cdot cm}. De gammastraling veroorzaakt een spanningsdaling over de elektrodes evenredig met hun weerstand. Aangezien de elektrodes zich gedragen als condensators, zorgt deze daling in spanning (gewoonlijk verwaarloosbaar bij afwezigheid van straling) ervoor dat het effectief elektrisch veld in het gasvolume bij een gegeven aangelegde spanning daalt, wanneer het aantal inkomende gammadeeltjes toeneemt. Wanneer we hier rekening mee houden verdwijnt deze verschuiving in de werkzame spanning. De overblijvende gemonitorde parameters zoals efficiëntie, de gemiddeld muonclustergrootte of de ladingsdepositie per ‘avalanche’, tonen geen enkel verschil tussen referentie- en bestraalde detector.\\
	Op dit moment is de CMS RPC groep nog steeds bezig om het huidige RPC-systeem te certifiëren voor de toekomstige HL-LHC periode. Deze langlevendheidstudie zou eind 2021 afgerond moeten zijn. Met de upgrade van het Link Board systeem zouden de huidige detectoren de hoge-luminositeitsfase van de LHC moeten kunnen doorstaan zonder grote veranderingen in hun performantie.

\clearpage{\pagestyle{empty}\cleardoublepage}


